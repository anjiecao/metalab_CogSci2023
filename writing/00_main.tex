% Template for Cogsci submission with R Markdown

% Stuff changed from original Markdown PLOS Template
\documentclass[10pt, letterpaper]{article}

\usepackage{cogsci}
\usepackage{pslatex}
\usepackage{float}
\usepackage{caption}

% amsmath package, useful for mathematical formulas
\usepackage{amsmath}

% amssymb package, useful for mathematical symbols
\usepackage{amssymb}

% hyperref package, useful for hyperlinks
\usepackage{hyperref}

% graphicx package, useful for including eps and pdf graphics
% include graphics with the command \includegraphics
\usepackage{graphicx}

% Sweave(-like)
\usepackage{fancyvrb}
\DefineVerbatimEnvironment{Sinput}{Verbatim}{fontshape=sl}
\DefineVerbatimEnvironment{Soutput}{Verbatim}{}
\DefineVerbatimEnvironment{Scode}{Verbatim}{fontshape=sl}
\newenvironment{Schunk}{}{}
\DefineVerbatimEnvironment{Code}{Verbatim}{}
\DefineVerbatimEnvironment{CodeInput}{Verbatim}{fontshape=sl}
\DefineVerbatimEnvironment{CodeOutput}{Verbatim}{}
\newenvironment{CodeChunk}{}{}

% cite package, to clean up citations in the main text. Do not remove.
\usepackage{apacite}

% KM added 1/4/18 to allow control of blind submission
\cogscifinalcopy

\usepackage{color}

% Use doublespacing - comment out for single spacing
%\usepackage{setspace}
%\doublespacing


% % Text layout
% \topmargin 0.0cm
% \oddsidemargin 0.5cm
% \evensidemargin 0.5cm
% \textwidth 16cm
% \textheight 21cm

\title{Pokemon go go go}


\author{Anjie Cao$^1$  (anjiecao@stanford.edu)
 and \bf{Michael C. Frank$^1$ (mcfrank@stanford.edu)} \\
$^1$Department of Psychology, Stanford University, }


\begin{document}

\maketitle

\begin{abstract}
HAHA

\textbf{Keywords:}
pikachu; mimikyu; ditto; jigglypufff
\end{abstract}

\hypertarget{introduction}{%
\section{Introduction}\label{introduction}}

haha!

In the first three years of life, children undergo a plethora of
developmental changes, transitioning from newborn infants who possess a
limited understanding of language and categories to toddlers who are
able to master a wide range of linguistic and cognitive skills. Despite
a wealth of research examining cognitive development, constructing a
comprehensive theory of cognitive development remains a formidable
challenge. Research in this area generally falls under two categories:
observational research and experimental research. The former provides a
holistic picture of an individual child's development (e.g.~CITE), yet
it does not provide any concrete insights into the underlying
mechanisms. In contrast, experimental research allows causal tractions
on potential mechanisms, but it tends to focus on one single construct
and does not reveal the connections between different processes and
mechanisms. In this paper, we aim to provide a quantitative synthesis of
experimental work across multiple areas of developmental psychology,
providing insights into the interrelatedness between psychological
constructs. We achieve this goal by consolidating and integrating 23
meta-analyses of cognitive and language development compiled on MetaLab,
a community-augmented meta-analysis platform.

Statistical meta-analysis, the technique of aggregating effect sizes
across a systematic sample of experiments, has some unique advantages as
a source of data about developmental processes in early childhood. First
and foremost, it allows researchers to explore questions that are
difficult to address with individual studies. One such example is the
functional form of developmental curves, or how different psychological
processes change over time. Many developmental studies use linear
regression models with age as a predictor, but this assumption of
linearity may not capture the complexities of developmental processes,
especially as they interact with developmental changes in measurement.
For example, some cognitive abilities -- such as relational reasoning --
might follow an inverted-U shape (Carstensen et al., 2019; Walkers,
Bridgers, \& Gopnik, 2016), while others -- like early vocabulary size
-- show an exponential increase (Frank, Braginsky, Yurovsky, \&
Marchman, 2021). These non-linear trends can be challenging to identify
and interpret with limited data from individual studies, but
meta-analytic methods can provide a large amount of data across a broad
age range, enabling researchers to evaluate and compare different
functional forms of developmental trajectories.

Meta-analysis can also shed light on the relationships between methods
and theories. Research methods and theories are fundamentally
intertwined, and this is especially true for developmental psychology
(Dale, Warlaumont, \& Johnson, 2022). Developmental theories are often
based on interpretations of experimental results, which are produced by
methods that even small changes to the parameters would substantially
change the outcomes. One example is the influence of familiarization
time. It has been proposed that the amount of exposure infants have
prior to the test events can influence infants' direction of preference
(i.e.~novelty preference or familiarity preference) (Hunter \& Ames,
1988). Although the empirical evidence for this theory is mixed, this
ambiguity has significant downstream consequences on our understanding
of infants' cognitive capabilities (Bergmann \& Cristia, 2016). Debates
about infants' arithmetic competencies or their evaluations of social
agents are often centered around the direction of preferences (Infants
arithmetic competencies: Clearfield \& Westfahl, 2006; Wakely, Rivera,
\& Langer, 2000; Wynn, 1992; Evaluation of social agents: Hamlin, Wynn,
\& Bloom; Salvadori et al., 2015). Due to the time and resources
required for developmental studies, it is often difficult to directly
evaluate the impact of subtle changes in parameters. Therefore,
meta-analytic methods provide a unique opportunity to investigate the
effects of methodological factors on research findings.

Last but not least, meta-analytic methods make it possible to compare
and connect research findings across research areas. The use of effect
size as the fundamental unit of analysis allows for comparisons across
different domains and research areas. These comparisons can provide
insight into how different processes facilitate learning at different
stages of development and can aid in the development of data-driven
cognitive development theories (Cao \& Lewis, 2021; Lewis et al., 2022).
However, a synthesis across multiple domains requires a database of
multiple meta-analyses. Towards that aim, MetaLab was established to
provide an open database of meta-analyses (CITE). Developmental
researchers are invited to deposit their meta-analysis dataset into
MetaLab, and they are encouraged to use the datasets for custom
analyses. As of DATE, Metalab contains X effect sizes from 30 different
meta-analyses. This resource would allow us to quantitatively synthesize
the insights across different research areas in developmental
psychology.

The plan for this paper is as follows. We first describe the datasets
included in the current synthesis, including our selection criteria and
the descriptive statistics associated with our final dataset. We then
turn to model comparison, comparing the fits of age models under
different functional forms. Next, we present methodological moderators
analysis. Four methodological moderators are selected due to their
theoretical relevances: behavioral measure type, exposure phase type,
stimuli type (audio and visual), and major author effect. Finally, we
present a synthesis of the developmental curves across all of the
domains considered. We end the paper by discussing the implications and
limitations of our current work.

\hypertarget{methods}{%
\section{Methods}\label{methods}}

gogogo

\hypertarget{datasets}{%
\subsection{Datasets}\label{datasets}}

\hypertarget{analytic-methods}{%
\subsection{Analytic Methods}\label{analytic-methods}}

\hypertarget{results}{%
\section{Results}\label{results}}

\hypertarget{functional-form-of-developmental-curves}{%
\subsection{Functional form of developmental
curves}\label{functional-form-of-developmental-curves}}

\hypertarget{methodological-moderators}{%
\subsection{Methodological Moderators}\label{methodological-moderators}}

\hypertarget{behavioral-measures}{%
\subsubsection{Behavioral Measures}\label{behavioral-measures}}

\hypertarget{exposure-phase}{%
\subsubsection{Exposure Phase}\label{exposure-phase}}

\hypertarget{stimuli-naturalness}{%
\subsubsection{Stimuli Naturalness}\label{stimuli-naturalness}}

\hypertarget{major-author}{%
\subsubsection{Major author}\label{major-author}}

\hypertarget{synthesis}{%
\subsection{Synthesis}\label{synthesis}}

lmao

\hypertarget{discussion}{%
\section{Discussion}\label{discussion}}

\hypertarget{references}{%
\section{References}\label{references}}

\setlength{\parindent}{-0.1in} 
\setlength{\leftskip}{0.125in}

\noindent

\bibliographystyle{apacite}


\end{document}
